\section{Produkteinsatz}

%\comment{Welche Anwendungsbereiche (Zweck), Zielgruppen (Wer mit welchen Qualifikationen), Betriebsbedingungen (Betriebszeit, Aufsicht)?}

\subsection{Anwendungsbereiche}
%Die Software soll w"ahrend Vorlesungen durch den Professor zu didaktischen Vorf"uhrungszwecken eingesetzt werden und von Studenten am Heimrechner zur Nacharbeitung verwendet werden k"onnen.

Die Software soll den Dozenten w"ahrend der Vorlesung und den Studenten im
Selbststudium unterst"utzen.

\subsection{Zielgruppen}
Zielgruppe sind Lehrende und Lernende, die die Veranschaulichung der Transformation von $C_0$ nach $H_0$ nutzen wollen.


\subsection{Betriebsbedingungen}
\begin{itemize}
  \item Das Modul l"asst sich auf einem Tablet-PC (mit Stift) bedienen.
  \item Der Dozent muss das Modul w"ahrend Vorlesung intuitiv bedienen k"onnen.
  \item Die Darstellung ist z.B. w"ahrend Vorlesungen durch den Beamer
  beeintr"achtigt.
  \begin{itemize}
    \item Bei der Darstellung s"amtlicher Inhalte wird auf die gr"o"sten Defizite durch den Beamer R"ucksicht genommen. (Unsch"arfe, niedrige Kontrastst"arke, falsche Farbdarstellung)
  \end{itemize}
\end{itemize}

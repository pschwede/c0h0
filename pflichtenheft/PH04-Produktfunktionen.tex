\section{Produktfunktionen}

%\comment{Was leistet das Produkt aus Benutzersicht?}

\subsection{Editor}
  \begin{description}
    \item[/F0110/]
      \textit{Programmcode anzeigen:} Der Editor ist in die grafische
      Oberfl"ache des Moduls integriert und enth"alt den aktuellen
      Programmcode.

    \item[/F0120/]
      \textit{Parserfehler anzeigen:} Parserfehler werden dem
      Benutzer auf der Bedienoberfl"ache angezeigt.

    \item[/F0121/]
      \textit{Code zum Editieren sperren und Algorithmus freischalten:}
      Diese Funktion soll Inkonsistenz zwischen Flussdiagramm, $H_0$-Code und
      dem zugrundeliegenden $C_0$-Code verhindern.

    \item[/F0122/]
      \textit{Code zum Editieren freischalten und Algorithmus zur"ucksetzen:}
      Erm"oglicht durch einen Klick das erneute Editieren des $C_0$-Codes.

    \item[/F0130/]
      \textit{Programmcode eingeben:}
      \begin{itemize}
        \item Bevor der Run-Button gedr"uckt wurde, kann der Benutzer den Code
        im Editor (\textbf{/F0110/}) "andern.
        \item Nachdem der Run-Button gedr"uckt wurde, wird der Programmcode
        entsprechend geparst (\textbf{/F0120/}) und anschlie"send formatiert.
        Anschlie"send ist kein Editieren mehr m"oglich.
      \end{itemize}

    \item[/F0140/]
      \textit{Programme laden:}
      \begin{itemize}
        \item In der Bedienoberfl"ache des Editors hat der Benutzer durch Dr"ucken des
        "Offnen-Buttons die M"oglichkeit, Programmcode aus einer Datei des Typs
        \textbf{/D010/} zu laden.
        In diesem Fall wird eine weitere Instanz des $C_0$$H_0$-Moduls ge"offnet.
				\item Der Inhalt der Datei erscheint gem"a"s \textbf{/F0110/}.
      \end{itemize}
    
    \item[/F0150/]
      \textit{Programme speichern:}
      \begin{itemize}
        \item Der Benutzer kann Programme, die sich nach \textbf{/F0110/} im
        Editor befinden, in einer $C_0$/$H_0$-Datei
        (\textbf{/D010/}, \textbf{/D020/}) abspeichern.
      \end{itemize}
  \end{description}
\newpage
\subsection{$C_0$/$H_0$-"Ubersetzung}
  Der Benutzer kann $C_0$-Programme in $H_0$-Programme "ubersetzen. Um
  diese Funktion zu nutzen, muss ein $C_0$-Programm im Editor stehen und der
  Run-Button gedr"uckt werden.
  \begin{description}
    \item[/F0210/]
      \textit{Programmcode anzeigen:} Dem Benutzer wird das aktuell in der
      Transformation befindliche Code-Statement in einem Textfeld hervorgehoben.
    \item[/F0220/]
      \textit{Anzeigen der Definition:} Die Definition der angewendeten
      Transformationsregel wird dem Benutzer angezeigt.
    \item[/F0230/]
      \textit{"Ubersetzungsschritt ausf"uhren:}
      \begin{description}
        \item[/F0231/]
        \textit{Baumstrukturierte Adressen generieren}
        \begin{itemize}
          \item Die Funktion wird ausgef"uhrt, indem der Einzelschritt-Button
          ''Vorw"arts'' geklickt wird. Dieser ist jedoch erst nach
          Ausf"uhrung der Sperrfunktion (\textbf{/F0121/}) m"oglich.
          \item Die Transformationsregel (\textbf{/F0220/}) des entsprechenden
          $C_0$-Programmabschnitts wird angezeigt.
          \item Die entsprechende baumstrukturierte Adresse erscheint auf der
          linken Seite des Code-Statements.
					Die Reihenfolge entspricht einer Tiefensuche, die Adressen werden also Zeile f"ur Zeile angezeigt.
          \item Der Benutzer kann somit nach jedem Transformationsschritt
          nachvollziehen, wie die Adresse entstanden ist.
          \item Man hat jederzeit die M"oglichkeit alle Schritte bis zum Ende
          der Teiltransformation ''Baumstrukturierte Adressen'' durchzuf"uhren,
          wodurch sofort alle Adressen dargestellt werden und das
          Flussdiagramm aufgebaut wird.  
        \end{itemize}

        \item[/F0232/]
        \textit{$C_0$ in Flussdiagramm transformieren}
        \begin{itemize}
          \item Die Funktion wird ausgef"uhrt, indem der Einzelschritt-Button
          ''Vorw"arts'' geklickt wird. Dies ist jedoch erst m"oglich, nachdem alle Baumstrukturierte Adressen generiert wurden.
          \item Das Flussdiagramm wird entsprechend des $C_0$-Codes aufgebaut.
					Dabei wird der Code-Baum durch Breitensuche abgelaufen.
          \item Man hat jederzeit die M"oglichkeit alle Schritte bis zum Ende
          der Teiltransformation ''Flussdiagramm'' durchzuf"uhren, wodurch sofort
          das Flussdiagramm aufgebaut wird.  
        \end{itemize}

	      \item[/F0233/]
        \textit{Abschlie"sende $H_0$ Transformation}
	      \begin{itemize}
	        \item Die Funktion wird ausgef"uhrt, indem der Einzelschritt-Button
          ''Vorw"arts'' geklickt wird. Dies ist jedoch erst m"oglich, nachdem
          das ''Flussdiagramm'' vollst"andig aufgebaut wurde.
          \item Es wird entsprechend des Skripts Schritt f"ur Schritt die
          abschlie"sende Transformation in den endg"ultigen $H_0$-Code
          durchgef"uhrt.
          \item Die jeweiligen Transformationsschritte und die dazu geh"origen trans-Funktionen werden dem Benutzer auf
          der Bedienoberfl"ache angezeigt.
          \item Man hat jederzeit die M"oglichkeit, alle Schritte bis zur
          endg"ultigen Transformation durchzuf"uhren.
	      \end{itemize}
      \end{description}
 \end{description}
    


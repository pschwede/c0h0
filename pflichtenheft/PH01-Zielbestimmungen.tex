\section{Zielbestimmungen}
Ziel der Arbeit dieser Projektgruppe ist die Entwicklung eines Moduls zur
Erweiterung des Lehrwerkzeugs j-Algo. Dieses Modul soll die schrittweise
Transformation von $C_0$-Code in $H_0$-Code entsprechend der Vorlesung
\textit{Grundlagen der Programmierung} veranschaulichen.

\subsection{Pflichtkriterien}

\begin{itemize}
	\subsubsection{Allgemein}
	\begin{itemize}
    \item Der Transformationsalgorithmus von $C_0$ in $H_0$ entspricht dem aus
    der Vorlesung \textit{Grundlagen der Programmierung} an der Technischen
    Universit"at Dresden.
    \item Der Algorithmus ist in Einzelschritte und zusammenfassende Schritte (''Baumstrukturierte Adressen'', ''Flussdiagramm'', ''$H_0$-Synthese'')
    gegliedert.
    \item Das Modul erm"oglicht es, oben genannten Algorithmus schrittweise
    ablaufen zu lassen.
    \item Jeder Schritt soll vom Modul anschaulich dargestellt werden.
    \item Zwischen den Einzelschritten kann der Nutzer vor und zur"uck
    springen.
    \item Der Nutzer hat jederzeit die M"oglichkeit, zum Ausgangsschritt
    zur"uckzukehren, um den $C_0$-Code zu editieren und den Algorithmus
    erneut zu beginnen.
  % \item In jeder Darstellung wird optimale Lesbarkeit (auch f"ur Vorlesungen)    garantiert. Es gibt keinen separaten Beamermodus.
		\item Schriftgr"o"sen lassen sich zwischen zwei Grundeinstellungen hin und her schalten,
		um f"ur den Heimgebrauch und die Vorlesung jederzeit optimale Lesbarkeit zu gew"ahrleisten. (Beamer-Modus)
    \item F"ur das Modul steht dem Nutzer eine Hilfestellung in Form eines
    Handbuchs zur Verf"ugung.
	\end{itemize}

	\subsubsection{$C_0$-Editor}
	\begin{itemize}
    \item Ein Texteditor ist in die grafische Oberfl"ache des Moduls
    integriert.
    \item Der Editor soll $C_0$-Statements entsprechend der $C_0$-Syntax
    farblich hervorheben.
    \item Bei der Erstellung eines neuen $C_0$-Programms soll das Modul
    folgende Parameter vom Benutzer erfragen: Anzahl der Variablen ($m$), Ein- ($k$) und Ausgabewerte ($i$).
		\item Der aus den Parametern erzeugte $C_0$-Code soll farblich in den Hintergrund treten.
    \item Der $C_0$-Code kann als c0-Datei gespeichert werden.
	  \item Der Editor unterst"utzt eine R"uckg"angig- und eine Wiederhol-Funktion.
	\end{itemize}

	\subsubsection{Validierung}
  \begin{itemize}
    \item Der $C_0$-Code ist per Mausklick validierbar.
    \item Ohne Validierung kann mit der Transformation nicht begonnen werden.
		\item Nach erfolgreicher Validierung des $C_0$-Codes wird der Code im Editor formatiert.
    \item Im Falle eines Syntaxfehlers wird der Nutzer auf diesen hingewiesen.
  \end{itemize}
  
  \subsubsection{Generierung baumstrukturierter Adressen}
	\begin{itemize}
    \item Der $C_0$-Code wird vor Beginn der Animation zur Generierung
    baumstrukturierter Adressen zur besseren Lesbarkeit ausgerichtet.
    \item Bei jedem Einzelschritt wird eine Adresse einem entsprechenden
    $C_0$-Statement zugeordnet.
    \item Der Bezug zwischen Adresse und entsprechendem $C_0$-Statement soll
    ersichtlich sein.
    \item Jedes $C_0$-Statement ist samt Adresse per Mausklick gesondert
    fokussierbar.
    \item Generierte Adressen sollen "ahnlich der Form `f121' sein.
    \item Generierte Adressen sollen  mit erkennbarem Bezug zum Codestatement
    im Editor zu sehen sein.
	\end{itemize}

	\subsubsection{Flussdiagramm}
	\begin{itemize}
    \item Ein Flussdiagramm ist in die grafische Oberfl"ache des Moduls
    integriert.
    \item Die Sichtbarkeit der Flussdiagramm-Abschnitte orientiert sich an den
    abgeschlossenen Transformationsschritten.
    \item Dabei soll der Bezug zum entsprechenden $C_0$-Code bzw. $H_0$-Code
    ersichtlich sein.
    \item Jeder Flussdiagramm-Abschnitt ist per Mausklick gesondert
    fokussierbar.
	\end{itemize}
	
	\subsubsection{$H_0$-Synthese}
	\begin{itemize}
    \item Bei jedem Einzelschritt wird ein $H_0$-Statement entsprechend dem
    oben genannten Algorithmus erzeugt und dargestellt.
    \item Dabei wird ebenfalls die gem"a"s des Algorithmus angewandte $trans$-Funktion angezeigt.
    \item "Ahnlich wie bei dem oben genannten Editor soll die $H_0$-Syntax
    farblich hervorgehoben werden.
    \item Dabei soll der Bezug zum entsprechenden $C_0$-Code bzw.
    Flussdiagramm-Abschnitt ersichtlich sein.
    \item Jedes $H_0$-Statement ist per Mausklick gesondert fokussierbar.
    \item Der $H_0$-Code kann als h0-Datei gespeichert werden.
	\end{itemize}
\end{itemize}

\subsection{Optionale Kriterien}
\begin{itemize}
  \item Das Flussdiagramm kann als g"angiges Grafikformat gespeichert
  werden.
	\item Mehrsprachigkeit
  \begin{itemize}
    \item Deutsch
    \item Englisch
  \end{itemize}
\end{itemize}

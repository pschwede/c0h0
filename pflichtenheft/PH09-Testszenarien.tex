\section{Testf"alle}

S"amtliche Produktfunktionen werden durch Testf"alle auf Korrektheit "uberpr"uft.

	\subsection{Allgemein}
	\begin{itemize}
		\item Im Praxistest mit einem Beamer wird die Farbdarstellung und die Lesbarkeit der Schrift "uberpr"uft.
		\item Das Programm wird mit verschiedenen Betriebssystemen, vor allem Windows und Linux getestet.
		\item Anhand verschiedener $C_0$-Testprogramme werden folgende Funktionen eingehend auf Korrektheit gepr"uft: Autoformatierung, baumstrukturierte Adressen, Schema, $H_0$-Code und Transformationsregeln.
	\end{itemize}
	
	\subsection{$C_0$-Editor / $H_0$-Ansicht}
	\begin{itemize}
    \item Erstellen eines neuen $C_0$-Codes \\
    $\Rightarrow$ Der Editor erscheint mit einem leeren $C_0$-Quelltextdokument.
		\item Laden einer $C_0$-Textdatei \\
		$\Rightarrow$ Der geladene Quelltext erscheint vollst"andig im Editor.
		\item Speichern des $C_0$/$H_0$-Codes \\
		$\Rightarrow$ Der Quelltext wird in eine $C_0$/$H_0$-Datei abgespeichert.
		\item Run-Button geklickt und $C_0$-Code ist fehlerhaft \\
		$\Rightarrow$ Bei der Validierung wird eine Fehlermeldung ausgegeben. \\
		$\Rightarrow$ Run-Button wird nicht zu Edit-Button
		\item Run-Button geklickt und $C_0$-Code ist fehlerfrei \\
		$\Rightarrow$ Der Code wird nach der Validierung automatisch formatiert. \\
		$\Rightarrow$ Run-Button wird zu Edit-Button
	\end{itemize}

	\item Umwandlungs-Modus
	
	\subsection{Interaktion}
	\begin{itemize}
		\item Klick auf einen der Einzelschritt-Button \\
		$\Rightarrow$ Ein Einzelschritt vorw"arts/r"uckw"arts wird entsprechend der algorithmischen Regel ausgel"ost.
		\item Klick auf Beginn-Button \\
		$\Rightarrow$ Ein vollst"andiger Arbeitsschritt wird r"uckg"angig gemacht, wobei alle entsprechenden Einzelschritte r"uckg"angig gemacht werden.
		\item Klick auf Schluss-Button \\
		$\Rightarrow$ Ein vollst"andiger Arbeitsschritt wird ausgef"uhrt, wobei alle entsprechenden Einzelschritte ausgef"uhrt werden.
		\item Klick auf Stelle im Code oder im Flussdiagramm \\
		$\Rightarrow$ Die entsprechende Stelle im $C_0$/$H_0$-Code, sowie im Flussdiagramm wird markiert. \\
		$\Rightarrow$ Die dazugeh"orige Transformationsregel wird angezeigt.
	\end{itemize}

\section{Produktleistungen}

	\subsection{Interaktivit"at}
	\begin{description}
		Der Benutzer hat die M"oglichkeit, die "Ubersetzung/Simulation schrittweise ablaufen zu lassen.
		Er kann Codeteile fokussieren, um mehr Informationen "uber diese Transformationsvorschrift zu erhalten.
		Die Schritte k"onnen durch den Benutzer r"uckg"angig gemacht werden.
	\end{description}

	\subsection{Ergonomie}
	\begin{description}
		Die Visualisierung wird f"ur die Darstellung am Beamer optimiert.
		Es wird dabei auf gro"se und kontrastreiche Darstellung geachtet.
	\end{description}

	\subsection{Fehlertoleranz}
	\begin{description}
		Eingegebene Programme werden auf Korrektheit gepr"uft.
		Bei fehlerhafter Eingabe wird dem Benutzer eine Meldung angezeigt.
		Der Benutzer hat daraufhin die M"oglichkeit seine Eingabe zu korrigieren.
	\end{description}

	\subsection{Fehlervermeidung}
	\begin{description}
		W"ahrend der "Ubersetzung/Simulation sind keine "Anderungen am jeweiligen Code m"oglich, um undefinierte Zust"ande zu vermeiden.
	\end{description}

	\subsection{Barrierefreiheit}
	\begin{description}
		Texte und Beschriftungen sind mehrsprachig (Deutsch und Englisch) verf"ugbar. (Optional)
		Die Module verf"ugen "uber eine Hilfe im HTML-Format.
	\end{description}

\end{itemize}


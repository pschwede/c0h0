\section{Der Arbeitsbereich}
\label{sec:DerArbeitsbereich}

Der Arbeitsbereich ist untergliedert in drei Bereiche, die Ihnen Zugriff auf alle wesentlichen Funktionen in den Algorithmus-Phasen erm�glicht:

\bigskip
\centerpic{kmp/phase1screen}{0.5}{Phase 1 Bildschirm}
\bigskip

   
\begin{enumerate}
	\item  \textbf{Steuerung}\newline
      Hier wird der Algorithmus gestartet und gesteuert.
   \item \textbf{Visualisierung}\newline
      Hier werden die Verschiebetabelle bzw. der Text dargestellt.
   \item \textbf{Dokumentation}\newline
      Hier gibt es verschiedene Perspektiven, die miteinander kombiniert werden k�nnen: 			Quellcode, Text, Protokoll.
\end{enumerate}

\section{Anzeigeoptionen}
\label{sec:Anzeigeoptionen}

\subsection{Skalierung}
\label{sec:Skalierung}

Die Gr��e der Elemente des Visualisierungs- und Dokumentationsbereichs kann eingestellt werden, klicken Sie daf�r auf den Schieberegler im Visualisierungsbereich und ziehen Sie ihn nach oben bzw. unten.

\centerpic{kmp/slide}{0.5}{Auswirkung des Schiebereglers zur Skalierung}
\bigskip
\subsection{Aufteilung der Bereiche}
\label{sec:Aufteilung der Bereiche}

Die Aufteilung zwischen dem Visualisierungs- und dem Dokumentationsbereich kann mit einem Schiebebalken ver�ndert werden. Per Klick auf die Kante k�nnen Sie die Grenze verschieben.

\centerpic{kmp/move}{0.5}{Auswirkung des Schiebereglers}
\bigskip

\subsection{Beamer - Modus}
\label{sec:BeamerModus}

Der Beamer-Modus erm�glicht das schnelle Einstellen von Anzeigeoptionen, die die Pr�sentation in Vorlesungen oder �hnlichen Veranstaltungen beg�nstigen. Dieser Modus ist zu finden unter 'Knuth Morris Pratt' => 'Beamermodus'.
Die Anzeige ist f�r die Aufl�sung 1024x768 optimiert und vergr��ert die Elemente um den Faktor 1,5. Im Dokumentationsbereich wird die Perspektive 'Code' angezeigt.
Ist der Modus aktiv, so erscheint vor diesem Men�eintrag ein H�kchen. Um den Modus wieder auszuschalten, entfernen Sie einfach den Haken per Klick.

\centerpic{kmp/beamermod}{0.5}{Der Beamermodus}
\bigskip
\subsection{Zyklus - Anzeige}
\label{sec:ZyklusAnzeige}

Im Visualisierungsbereich der Phase 1 'Generierung der Verschiebetabelle' k�nnen optional die Zyklen angezeigt werden, setzen Sie dazu das H�kchen 'Zyklen' im Steuerungsbereich. Es werden maximal drei Zyklen gleichzeitig angezeigt.

\centerpic{kmp/cycle}{0.5}{Beispiel f�r Zyklen}
\bigskip
\section{Legende}
\label{sec:Legende}

\subsection{Elemente in der Phase 'Generierung der Verschiebetabelle'}
\label{sec:ElementeInDerPhaseGenerierungDerVerschiebetabelle}

\bigskip
\centerpic{kmp/phase1screen_ex1}{0.5}{Beispiel f�r Phase 1}
\bigskip
\textit{Visualisierungsbereich}\newline
\textbf{Pfeil schwarz mit wei�em 'P' }- Zeiger auf die Variable 'patpos', die Patternposition\newline
\textbf{Pfeil wei� mit schwarzem 'V' }- Zeiger auf die Variable 'VglInd', der Vergleichsindex\newline
\textbf{schwarzer Pfeil �ber der Tabelle }- die verglichenen Zeichen\newline
\textbf{gelber Hintergrund von Zellen} - Zellenkopf\newline
\textbf{schwarzer Rahmen um Zelle in der Zeile 'Index' }- aktuell kopierte Verschiebeinformation\newline
\textbf{roter Rahmen um Zellen }- negative boolesche Bedingung\newline
\textbf{lila Strich }- Zyklen, die im Pattern auftreten
\newline\newline
\textit{Schriftfarben}\newline
\textbf{blau} - aktuell geschriebene Verschiebeinformation (entspricht Zuweisungsstatement im Code)\newline
\textbf{rot} - die verglichenen Zeichen stimmen nicht �berein (entspricht booleschem Statement im Code)\newline
\textbf{gr�n} - die verglichenen Zeichen stimmen �berein (entspricht booleschem Statement im Code)

\subsection{Elemente in der Phase 'Suchen im Suchtext'}
\label{sec:ElementeInDerPhaseSuchenImSuchtext}
 \bigskip
\centerpic{kmp/phase2screen_ex1}{0.5}{Beispiel f�r diese Phase}
 \bigskip
\textit{Visualisierungsbereich}\newline
\textbf{lila Rahmen }- das Sichtfenster, welches die aktuell betrachteten Zeichen des Textes und des Patterns justiert
\newline\newline
\textit{Schriftfarben}\newline
\textbf{blau} - aktuell geschriebene Verschiebeinformation (entspricht Zuweisungsstatement im Code)

\subsection{Im Dokumentationsbereich}
\label{sec:ImDokumentationsbereich}

\bigskip
\centerpic{kmp/phase2screen_ex2}{0.5}{Beispiel f�r diese Phase}
\bigskip
\textit{Perspektive 'Code'}\newline
\textbf{roter Hintergrund }- negative boolesche Bedingung\newline
\textbf{gr�ner Hintergrund} - positive boolesche Bedingung\newline
\textbf{blauer Hintergrund }- sonstige Statements
\newline\newline
\textit{Perspektive 'Protokoll'}\newline
\textbf{blaue Schrift} - zuletzt vollzogener Schritt\newline
\textbf{schwarze Schrift} - f�r den aktuellen Stand irrelevante vorausgegangene Schritte
\newline\newline
\textit{Perspektive 'Suchtext'}\newline
\textbf{gelber Hintergrund} - der Textausschnitt, an dem das Pattern momentan anliegt

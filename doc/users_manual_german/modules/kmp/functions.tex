\section{Modulfunktionen}
\label{sec:Modulfunktionen}

\subsection{Eingabe von Pattern und Text}
\label{sec:EingabeVonPatternUndText}

\subsubsection{Pattern eingeben}
\label{sec:PatternEingeben}


\centerpic{kmp/pattern1}{0.5}{Pattern manuell eingeben}
\bigskip
Das Pattern kann manuell eingegeben werden, hierbei ist die maximale L�nge von 10 Zeichen zu beachten. Das Alphabet kann aus beliebigen Zeichen bestehen. Die Eingabe erfolgt im Steuerungsbereich in die Eingabezeile 'Pattern'. Um das Pattern zu �bernehmen klicken Sie auf 'Setzen'.

\subsubsection{Suchtext eingeben}


Der Text unterliegt keinen Beschr�nkungen und kann manuell eingegeben oder aus einer vorhandenen *.txt-Datei importiert werden.
Die Eingabe erfolgt im Steuerungsbereich, klicken Sie hierf�r auf 'Eingeben'.

\centerpic{kmp/text1}{0.5}{Text manuell eingeben}
\bigskip
Es �ffnet sich ein Fenster, in dem Sie den Text manuell eingeben (oder auch per Copy und Paste einf�gen) oder eine *.txt-Datei laden k�nnen. Um den Text zu �bernehmen klicken Sie auf 'Anwenden'.

\centerpic{kmp/text1}{0.5}{Text laden oder eingeben}
\bigskip
\subsection{Generieren von Pattern und passenden Texten}
\label{sec:GenerierenVonPatternUndPassendenTexten}

\subsubsection{Pattern eingeben}

\centerpic{kmp/pattern2}{0.5}{Pattern generieren lassen}
\bigskip
Neben der manuellen Eingabe des Patterns gibt es die M�glichkeit, ein Pattern generieren zu lassen. Klicken Sie daf�r auf den Knopf 'Zufall' (im Steuerungsbereich neben der Eingabezeile 'Pattern') und w�hlen Sie im folgenden Fenster die Kardinalit�t des zu nutzenden Alphabets und die gew�nschte L�nge des Patterns aus. Das Pattern wird mit Klick auf 'Anwenden' geniert und Sie kehren zum Arbeitsbereich zur�ck.

\centerpic{kmp/pattern3}{0.5}{Pattern generieren lassen}
\bigskip
\subsubsection{Suchtext eingeben}

\centerpic{kmp/text3}{0.5}{Suchtext generieren lassen}
\bigskip
Um einen zum Pattern passenden Text generieren zu lassen, klicken Sie auf 'Generieren' (im Steuerungsbereich neben 'Text'). Dadurch wird eine Text erstellt, der das Pattern enth�lt und nur aus den im Pattern genutzten Zeichen besteht. Dieser Text wird sofort �bernommen, Sie haben aber die M�glichkeit, ihn zu bearbeiten, indem Sie auf 'Eingeben' klicken.


\subsection{Generierung der Verschiebetabelle}
\label{sec:GenerierungDerVerschiebetabelle}


Wenn ein Pattern gesetzt wurde, kann die Verschiebetabelle erstellt werden. Um den Algorithmus zu starten und zu steuern, benutzen Sie die Pfeile im Steuerungsbereich.

\centerpic{kmp/control}{0.5}{Algorithmussteuerung}
\bigskip
Das Pattern kann nun noch erweitert werden ohne den aktuellen Algorithmus zu unterbrechen, geben Sie hierf�r das gew�nschte Zeichen in das Eingabefeld "Erweiterung" im Steuerungsbereich ein. Sollte das Pattern bereits die maximale L�nge haben, ist eine Erweiterung nicht m�glich.

\centerpic{kmp/expand}{0.5}{Pattern erweitern}
\bigskip
Nachdem die Verschiebetabelle erstellt wurde, k�nnen Pattern und Verschiebetabelle f�r eine Suche im Text genutzt werden. Klicken Sie hierf�r auf das Feld 'Weiter zu Phase 2', welches erst am Ende des Algorithmus im Steuerungsbereich erscheint.

\centerpic{kmp/go_p2}{0.5}{Von Phase 1 in Phase 2 wechseln}
\bigskip
\subsection{Suchen im Suchtext}
Erst wenn Pattern und Text gesetzt sind, kann das Anwenden der Verschiebetabelle an einem Text durchgef�hrt werden. Um Pattern und Text zu setzen, haben Sie verschiedene M�glichkeiten, siehe: Eingabe von Pattern und Text , Generieren von Pattern und passenden Texten.
Um den Algorithmus zu starten und zu steuern, benutzen Sie die Pfeile im Steuerungsbereich.

\centerpic{kmp/control}{0.5}{Algorithmussteuerung}
\bigskip
Sollten Sie w�hrend der Durchf�hrung des Algorithmus das Pattern und/oder den Text �ndern, wird die aktuelle Durchf�hrung des Algorithmus unterbrochen.
\bigskip
Der Suchvorgang wird beendet, wenn das Pattern gefunden wurde oder das Textende erreicht wurde, es erscheint eine Meldung dazu im Steuerungsbereich.

\centerpic{kmp/endnegativ}{0.5}{Nachricht, dass Pattern nicht gefunden wurde}
\bigskip
\centerpic{kmp/endpositiv}{0.5}{Nachricht, dass Pattern gefunden wurde}
\bigskip
\subsection{�ffnen einer KMP-Sitzung}
\label{sec:�ffnenEinerKMPSitzung}


Mit Klick auf das Ordner-Symbol �ffnet sich ein Dialogfenster, in dem Ihnen die M�glichkeit gegeben wird, eine '*.jalgo' - Datei auszuw�hlen, in welcher eine KMP-Sitzung gespeichert wurde.

\centerpic{kmp/load}{0.5}{Dialogfenster zum Laden einer Sitzung}
\bigskip

\subsection{Pr�sentation von Lernbeispielen}
Es stehen Ihnen zehn Beispiele zur Auswahl, die jeweils eine besondere Eigenschaft des KMP-Algorithmus repr�sentieren. Wenn Sie mit dem Mauszeiger �ber die Beispiele fahren, werden diese Eigenschaften kurz beschrieben. W�hlen Sie das gew�nschte Lernbeispiel aus und klicken Sie auf 'Laden' um das Beispiel zu starten.\newline
Die Beispiele bestehen aus Pattern- und passendem Text.

\centerpic{kmp/learning}{0.5}{Pr�sentation von Lernbeispielen}
\bigskip
Die Steuerung der Lernbeispiele erfolgt durch die Pfeile im Steuerungsbereich.


\chapter{Das Hauptprogramm}
\section{Einleitung}
Dieses Handbuch stellt eine Einf�hrung und Hilfe f�r die Arbeit mit \jalgo dar. \jalgo ist eine Software, die sich mit der Visualisierung von Algorithmen besch�ftigt. Sie soll dazu dienen, verschiedene Algorithmen zu veranschaulichen um sie so Studenten und anderen Interessierten verst�ndlicher zu machen. Die Anwendung basiert auf einer Plugin-Struktur, die es erm�glicht, einzelne Module, die jeweils einen Algorithmus oder ein Themengebiet abdecken k�nnen, in das Programm zu integrieren und zu laden.

Sowohl \jalgo als auch die einzelnen Module entstanden im Rahmen des externen Softwarepraktikums im Studiengang Informatik der TU Dresden in Zusammenarbeit mit dem Lehrstuhl Programmierung. Die implementierten Module orientieren sich daher an den Lehrveranstaltungen "`Algorithmen und Datenstrukturen"' sowie "`Programmierung"' im Grundstudium Informatik an der TU Dresden. Das Einsatzgebiet soll vor allem die Vorlesung und das studentische Lernen zu Hause umfassen.\\
\jalgo ist eine freie Software, die beliebig oft kopiert werden darf.

\bigskip
\section{Technische Hinweise}
\subsection{Systemvoraussetzungen}
Folgende minimale Systemanforderungen werden f�r den reibungslosen Einsatz von \jalgo ben�tigt:
\begin{itemize}
	\item IBM-kompatibler PC 
	\item Mindestens 64 MB RAM
	\item {\sc Windows} 98(SE)/ME/2000/XP , {\sc Linux} SuSE/Red Had 	
	\item Java 2 Platform Standard Edition 5.0 {\small (siehe: \href{http://java.sun.com/}{http://java.sun.com/})}
	\item Maus und Tastatur	
	\item Monitor mit einer Aufl�sung von mindestens 800x600 
\end{itemize}

\medskip
\subsection{Installation}
\subsubsection*{Windows}
Entpacken Sie nach dem Herunterladen das ZIP-komprimierte Archiv in einen Ordner Ihrer Wahl. 
In diesem Ordner finden Sie eine Datei namens "`j-algo.bat"'. 
�ffnen Sie diese Datei mit einem Doppelklick, und das Programm wird gestartet.

\subsubsection*{Unix}
Entpacken Sie nach dem Herunterladen das TGZ-komprimierte Archiv in einen Ordner Ihrer Wahl. 
In diesem Ordner finden Sie eine Datei namens "`j-algo.sh"'. 
�ffnen Sie die Konsole und starten sie mittles \verb|sh j-algo.sh| das Programm.

\medskip
\subsection{Deinstallation}
Der komplette Programmordner kann jederzeit gefahrlos von der Festplatte gel�scht werden.

\newpage
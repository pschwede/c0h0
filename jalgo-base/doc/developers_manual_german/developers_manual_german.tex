\documentclass{scrartcl}
\usepackage[ngerman]{babel}
\usepackage[T1]{fontenc}
\usepackage[ansinew]{inputenc}
\usepackage{a4wide}

%verwenden von grafiken
\usepackage[dvipdf, final]{graphicx}

%verwenden von hyperlinks, stil derselben
\usepackage{color}
\definecolor{darkblue}{rgb}{0,0,0.5}

\usepackage{hyperref}
\hypersetup{
%	draft,										%hyperlinks ausschalten				
	colorlinks,									%hyperlinks farbig darstellen
	linkcolor   = darkblue,
	filecolor   = darkblue,
	urlcolor    = darkblue,
	citecolor   = darkblue,
	pdftitle    = {Entwicklerhandbuch},		%titel
	pdfsubject  = {j-Algo},			%thema
	pdfauthor   = {Alexander Claus, Matthias Schmidt},
	pdfkeywords = {Algorithmen, Visualisierung},
	pdfcreator  = {Distiller},
	pdfproducer = {LaTeX mit Hyperref-Package}}

%verwenden von code-ausschnitten
%\usepackage{listings}

%spezielle kommandos
%schreibweise des software-titels
\newcommand{\jalgo}{\mbox{\bfseries {\color{red}j}-Algo} }
%pfad zu den bildern
\newcommand{\pfad}{pics/}
%f�gt ein bild an einer bestimmten stelle, relativ zur position des befehls, ein.
%usage: \icon{dateiname}{y-offset}{x-offset}{bildvergr��erung}
\newcommand{\icon}[4]{
	\vspace{#2 ex}
	\hspace{#3 ex}
	\includegraphics[scale=#4]{\pfad #1}
}
%f�gt ein bild mittig mit bildunterschrift ein.
%usage: \centerpic{dateiname}{bildvergr��erung}{untertitel}
\newcommand{\centerpic}[3]{
	\begin{center}
		\includegraphics[scale=#2]{\pfad #1}\\
		{\small #3}
	\end{center}
}
%f�gt eine \subsection mit einem f�hrenden icon ein
%usage: \subsectionicon{text}{icon}
\newcommand{\subsectionicon}[2]{
	\subsection[#1]{\qquad #1}
	\icon{#2}{-5}{7}{1}
	\\
}
%f�gt eine \subsection mit 2 f�hrenden icons ein
%usage: \subsectiondoubleicon{text}{icon1}{icon2}
\newcommand{\subsectiondoubleicon}[3]{
	\subsection[#1]{\qquad \quad #1}
	\icon{#2}{-5}{7}{1}
	\icon{#3}{0}{-2}{1}
	\\
}
%f�gt eine \subsubsection* mit 2 f�hrenden icons ein
%usage: \subsubsectiondoubleicon{text}{icon1}{icon2}
\newcommand{\subsubsectiondoubleicon}[3]{
	\subsubsection*{\qquad \qquad #1}
	\icon{#2}{-4}{1}{1}
	\icon{#3}{0}{-2}{1}
	\\
}

\begin{document}

\begin{titlepage}
\centerpic{title}{1}{}
\vfill
\begin{flushright}
{\Huge \textbf{Entwicklerhandbuch}}
\end{flushright}
\end{titlepage}

\newpage

\tableofcontents
\newpage

\section{Einleitung}
Dieses Handbuch soll k�nftigen Entwicklern von \jalgo helfen, sich schnell mit der Struktur der Software auseinanderzusetzen. \jalgo ist eine Software, die sich mit der Visualisierung von Algorithmen besch�ftigt. Sie soll dazu dienen, verschiedene Algorithmen zu veranschaulichen um sie so Studenten und anderen Interessierten verst�ndlicher zu machen. Die Anwendung basiert auf einer Plugin-Struktur, die es erm�glicht, einzelne Module, die jeweils einen Algorithmus oder ein Themengebiet abdecken k�nnen, in das Programm zu integrieren und zu laden.

Sowohl \jalgo als auch die einzelnen Module entstanden im Rahmen des externen Softwarepraktikums im Studiengang Informatik der TU Dresden in Zusammenarbeit mit dem Lehrstuhl Programmierung. Die implementierten Module orientieren sich daher an den Lehrveranstaltungen "`Algorithmen und Datenstrukturen"' sowie "`Programmierung"' im Grundstudium Informatik an der TU Dresden. Das Einsatzgebiet soll vor allem die Vorlesung und das studentische Lernen zu Hause umfassen.\\
\jalgo ist eine freie Software, die beliebig oft kopiert werden darf.

\bigskip
\section{Technische Hinweise}
\subsection{Systemvoraussetzungen}
Folgende minimale Systemanforderungen werden f�r den reibungslosen Einsatz von \jalgo ben�tigt:
\begin{itemize}
	\item IBM-kompatibler PC 
	\item Mindestens 64 MB RAM
	\item {\sc Windows} 98(SE)/ME/2000/XP , {\sc Linux} SuSE/Red Had 	
	\item Java 2 Platform Standard Edition 5.0 {\small (siehe: \href{http://java.sun.com/}{http://java.sun.com/})}
	\item Maus und Tastatur	
	\item Monitor mit einer Aufl�sung von mindestens 800x600 
\end{itemize}

\medskip
\subsection{Installation}
\subsubsection*{Windows}
Entpacken Sie nach dem Herunterladen das ZIP-komprimierte Archiv in einen Ordner Ihrer Wahl. 
In diesem Ordner finden Sie eine Datei namens "`j-algo.bat"'. 
�ffnen Sie diese Datei mit einem Doppelklick, und das Programm wird gestartet.

\subsubsection*{Unix}
Entpacken Sie nach dem Herunterladen das TGZ-komprimierte Archiv in einen Ordner Ihrer Wahl. 
In diesem Ordner finden Sie eine Datei namens "`j-algo.sh"'. 
�ffnen Sie die Konsole und starten sie mittles \verb|sh j-algo.sh| das Programm.

\medskip
\subsection{Deinstallation}
Der komplette Programmordner kann jederzeit gefahrlos von der Festplatte gel�scht werden.

\newpage
\section{CVS-Zugang}
\jalgo ist als Projekt bei \href{https://sourceforge.net/projects/j-algo/}{SourceForge} registriert und gehostet. Es gibt 2 Arten, auf das CVS-Repository des Projektes zuzugreifen.
\begin{enumerate}
	\item Lesezugriff. Als Beobachter des Projektes kann jeder auf das Projekt lesend zugreifen. Die Zugangsdaten sind:\\
		Verbindungsmethode: \verb|pserver|\\
		Host: \verb|cvs.sourceforge.net|\\
		Repository-Pfad: \verb|/cvsroot/j-algo|\\
		Login: \verb|anonymous|
	\item Vollzugriff. Als registrierter Entwickler bei SourceForge und als eingetragenes Projektmitglied bei \jalgo kann das CVS unter folgenden Zugangsdaten im Vollzugriff erreicht werden:\\
		Verbindungsmethode: \verb|extssh|\\
		Host: \verb|cvs.sourceforge.net|\\
		Repository-Pfad: \verb|/cvsroot/j-algo|\\
		Login: <\textsc{SourceForge-Login}>\\
		Passwort: <\textsc{SourceForge-Passwort}>\\
		Um als Projektmitglied eingetragen zu werden, wenden Sie sich bitte an den Projekt-Administrator. Dessen Kontaktdaten sind auf der SourceForge-Seite zug�nglich.
\end{enumerate}
\textbf{Achtung:} Wird das Projekt im Rahmen des Software-Praktikums an der TUD weiterentwickelt, gelten andere Bedingungen f�r den CVS-Zugang. Diese sind beim zust�ndigen Betreuer des Praktikums zu erfragen.

\newpage

\section{Entwickeln unter Eclipse}
Nat�rlich steht es jedem Entwickler frei, eine Programmierumgebung seiner Wahl zu benutzen. Da jedoch der Gro�teil der \jalgo-Entwickler unter \href{http://eclipse.org}{Eclipse} programmiert, und diese Plattform einige komfortable Features besitzt, sollen hier die wichtigsten Einstellungen f�r diese Umgebung erl�utert werden. F�r andere Programmierumgebungen gelten sie sinngem��.\\
Da \jalgo die Java-Version 1.5 verwendet, ist eine Eclipse-Version 3.1 oder h�her erforderlich.\footnote{Achtung, auf der Seite des Eclipse-Projektes wird noch auf die Version 1.4.2 des JDK verlinkt}

Das Projekt kann in der CVS-Ansicht von Eclipse ausgecheckt werden. Ab jetzt sind zwar alle n�tigen Daten (Quellcodes, etc.) auf dem Rechner. Allerdings m�ssen noch einige Einstellungen vorgenommen werden, damit das Projekt kompiliert und gestartet werden kann:
\begin{itemize}
	\item Unter den Projekteinstellungen->"'Java Compiler"'->"`Compiler Compliance Level"' ist "`5.0"' einzustellen. Es ist sicherzustellen, dass unter Projekteinstellungen->Libraries als "`JRE System Library"' die Version 1.5 eingestellt ist. Ist dies nicht der Fall, so ist diese mittels "`Add Library..."'->"`JRE System Library"' einzustellen.
	\item Unter Projekteinstellungen->Info->Text file encoding muss UTF-8 eingestellt werden. Dies garantiert reibungslose Unterst�tzung von Umlauten auf verschiedenen Betriebssystemen.
	\item Unter Projekteinstellungen->Java Build Path m�ssen jetzt einige Einstellungen f�r den ClassPath des Projektes vorgenommen werden:\\
Unter Source darf nur der Ordner \textsc{<Projektordner>}\verb|/src| stehen. Ist dies nicht der Fall, ist dieser mittels "`Add Folder..."' aus der Ordnerliste auszuw�hlen. Andere Ordner sind zu entfernen.
	\item Als "`Default Output Folder"' ist \textsc{<Projektordner>}\verb|/bin| anzugeben.
	\item Unter "`Libraries"'->"`Add JARs..."' ist \textsc{<Projektordner>}\verb|/extlibs/jh.jar| hinzuzuf�gen. Dies ist die n�tige Bibliothek f�r das Hilfe-System.
	\item Unter "`Libraries"'->"`Add Class Folder..."' sind folgende Ordner hinzuzuf�gen:\\
		\textsc{<Projektordner>}\verb|/runtime| (f�r die Erkennung der installierten Module)\\
		\textsc{<Projektordner>}\verb|/res/main| (f�r die Ressourcen zum Hauptprogramm)\\
		sowie alle verf�gbaren Modulordner unter
		\textsc{<Projektordner>}\verb|/res/module/|, also z.B.
		\textsc{<Projektordner>}\verb|/res/module/testModule| (f�r die Ressourcen der einzelnen Module)
	\item Als n�chstes werden die jUnit-Bibliotheken ben�tigt. Weil Eclipse diese bereits eingebaut hat, ist die einfachste Variante, diese hinzuzuf�gen, folgenderma�en:\\
Projekteinstellungen �bernehmen, Workspace neu kompilieren lassen, und dann im View "`Problems"' einen der vielen Fehler ausw�hlen, die im Zusammenhang mit jUnit gebracht werden. Beim �ffnen des gew�hlten Source-Files zeigt Eclipse im Editor bei den entsprechenden Imports Fehler an. Dr�cken Sie genau dort auf das rote Kreuz, und Ihnen wird die Option angeboten "`Add jUnit libraries"'. W�hlen Sie diese aus, schlie�en das Source-File, und fertig.\\
Jetzt sollte im View "`Problems"' kein Fehler mehr angezeigt werden.
	\item Nun muss noch eine Startkonfiguration erstellt werden, und dann sind wir fertig:\\
Unter dem Men�punkt Run->Run... erstellen Sie eine neue Konfiguration vom Typ "`Java Application"', vergeben einen sinnvollen Namen und w�hlen vom \jalgo-Projekt als "`Main-Class"' \verb|org.jalgo.main.JAlgoMain| aus.
\end{itemize}

Jetzt ist das Projekt kompilierbar und das Programm kann gestartet werden.

\section{Projektstruktur}
Es folgt ein kurzer �berblick �ber die bestehende Struktur des Projektes, so dass der Entwickler wei�, welche Teile er ver�ndern darf, und welche besser unangetastet bleiben sollten\dots\\
Das Projekt fasst mehrere Ordner und einige "`lose"' Dateien. Der Reihe nach:
\begin{itemize}
	\item Der Ordner \verb|bin| fasst die kompilierten Klassen. Sein Inhalt kann gel�scht werden, er wird bei jedem kompilieren neu erstellt. (Hinweis: Dieser Ordner geh�rt nicht unter die Versionskontrolle!)
	\item Der Ordner \verb|doc| fasst die Projektdokumentation. Dies sind die Dateien zum Entwicklerhandbuch, zum Benutzerhandbuch, sowie einige Dateien, die gewisse aufgetretene Probleme und evtl. Abhilfen schildern.
	\item Im Ordner \verb|examples| sind Beispieldateien f�r jedes Modul enthalten. Der komplette Ordner wird sp�ter in der Distribution enthalten sein.
	\item Im Ordner \verb|extlibs| liegen Bibliotheken, die Fremdcode enthalten. Dies ist derzeit nur die Laufzeitbibliothek des Hilfesystems. Der komplette Ordner wird sp�ter in der Distribution enthalten sein.
	\item Der Ordner \verb|relicts| fasst Codeteile und Ressourcendateien, welche derzeit nicht mehr verwendet werden. Sie wurden trotzdem aufgehoben, weil sie teilweise Funktionalit�t enthalten, die zu implementieren mal begonnen wurde, die jedoch nie ausgereift waren und daher derzeit nicht verwendet werden. Vielleicht bringen Sie einen Nutzen, wenn der Entwickler Ideen sucht.
	\item Im Ordner \verb|res| liegen alle Ressourcendateien geordnet nach Programmteilen.
	\item Der Ordner \verb|runtime| enth�lt leere, aber notwendige Dateien f�r die Laufzeit. Sie sind Teil der Pluginstruktur, und erm�glichen das Erkennen der installierten Module.
	\item Im Ordner \verb|src| schlie�lich ist der Quellcode enthalten. Die Paketstruktur ist intuitiv gehalten. Unter \verb|org.jalgo.main| findet sich alles, was zum Hauptprogramm geh�rt, und unter \verb|org.jalgo.module| liegen alle Modulpakete.
	\item Die "`losen"' Dateien sind diverse Build-Skripte, Manifest- und Start-Dateien f�r verschiedene Betriebssysteme sowie einige projektspezifische Dateien.
\end{itemize}
Teilweise wird in diesem Entwicklerhandbuch auf die API-Dokumentation von \jalgo verwiesen. Da an der Software permanent gearbeitet wird, ist diese nicht unter der Versionskontrolle verf�gbar, sondern sollte vom Entwickler selbst in regelm��igen Abst�nden generiert werden. Arbeitet der Entwickler unter Eclipse, so kann dies unter dem Men�punkt Project->"`Generate Javadoc..."' ganz einfach durchgef�hrt werden.
\newpage
\newcommand{\code}[1]{\lstinline$#1$}

\section{Implementieren eines neuen Moduls}
Hier sollen nur die technischen Schritte angegeben sein, die n�tig sind, ein neues Modul f�r \jalgo korrekt und vollst�ndig aufzusetzen und in das Hauptprogramm zu integrieren. Es wird vorausgesetzt, dass der Entwickler selbst ein Konzept seines Moduls entwickelt, insbesondere, was Details der Visualisierung betrifft.

Es folgen f�nf Abschnitte. Im ersten wird der Teil erkl�rt, der f�r das Implementieren eines neuen Moduls minimal notwendig ist. Der zweite Abschnitt enth�lt Erkl�rungen zur Funktionsweise der Pluginstruktur von \jalgo, und was der Entwickler f�r korrekte Erkennung des Moduls zu beachten hat. Der dritte Abschnitt zeigt, wie die Ressourcen des zu implementierenden Moduls organisiert sein sollten. Der vierte Abschnitt erkl�rt, wie die Dateien der Online-Hilfe f�r das Modul zu organisieren sind und der f�nfte Abschnitt schlie�lich zeigt die hauptprogrammseitige Schnittstelle zwischen Modul und Hauptprogramm. Im folgenden wird abk�rzend f�r "`Das zu implementierende Modul"' nur "`Das Modul"' geschrieben.

\subsection{Grundimplementierung}
Der Code f�r das Modul wird im Paket \verb|org.jalgo.module.|\textsc{<Modulk�rzel>} abgelegt. Das Modulk�rzel sollte aussagekr�ftig, jedoch relativ kurz gehalten sein. Es wird sp�ter im Code oft ben�tigt, wenn es um Ressourcen-Zugriffe geht.\\
Eine schlechte Wahl w�ren also zum Beispiel \verb|dijkstrasShortestPathAlgorithm| oder vielleicht \verb|syntaxDiagramsAndEBNF|.

Es gibt zwei Schnittstellen, um das Hauptprogramm mit einem Modul zu vernetzen. Die erste, modulseitige, Schnittstelle besteht aus zwei Klassen. Jedes Modul muss eine Verbindungseinheit und eine Informationseinheit anbieten. Die Verbindungseinheit muss abgeleitet sein von \verb|org.jalgo.main.AbstractModuleConnector|. Hier sind Methoden zu implementieren, die die Interaktion des Moduls mit dem Hauptprogramm spezifizieren. F�r Details dazu sollte die API-Dokumentation von \jalgo konsultiert werden. Nachfolgend ist die Schnittstelle von \verb|AbstractModuleConnector| abgebildet.\\[0.25cm]
\textbf{Achtung!} Die Verbindungseinheit muss sich an eine Namenskonvention halten: Paket und Name der Klasse muss \verb|org.jalgo.module.|\textsc{<Modulk�rzel>}\verb|.ModuleConnector| lauten. Dies ist ein notwendiger Teil der Pluginstruktur von \jalgo. F�r Details dazu lesen Sie bitte den n�chsten Abschnitt.
\newpage
\begin{verbatim}
public abstract class AbstractModuleConnector {
    public abstract void init();
    public abstract void run();
    public abstract void setDataFromFile(ByteArrayInputStream data);
    public abstract ByteArrayOutputStream getDataForFile();
    public abstract void print();

    public boolean close();

    public final IModuleInfo getModuleInfo();
    public enum SaveStatus {
        NOTHING_TO_SAVE,
        NO_CHANGES,
        CHANGES_TO_SAVE
    }
    public final SaveStatus getSaveStatus();
    public final void setSaveStatus(SaveStatus status);
    public final void setSavingBlocked(boolean blocked);
    public final boolean isSavingBlocked();
    public final String getOpenFileName();
    public final void setOpenFileName(String filename);
}
\end{verbatim}

Die Informationseinheit ist dazu da, Informationen �ber das Modul bereitzustellen, die dem Benutzer entsprechend aufbereitet dargeboten werden, wenn er ein Modul zur Benutzung ausw�hlen will. Sie muss das Interface \verb|org.jalgo.main.IModuleInfo| implementieren. F�r Details dazu sei hier wieder auf die API-Dokumentation von \jalgo verwiesen. Nachfolgend ist die Schnittstelle von \verb|IModuleInfo| abgebildet.
\begin{verbatim}
public interface IModuleInfo {
    public String getName();
    public String getVersion();
    public String getAuthor();
    public String getDescription();
    public URL getLogoURL();
    public String getLicense();
    public URL getHelpSetURL();
}
\end{verbatim}
\textbf{Achtung!} Die Informationseinheit muss sich an eine Namenskonvention halten: Name und Paket der Klasse muss \verb|org.jalgo.module.|\textsc{<Modulk�rzel>}\verb|.ModuleInfo| lauten. Dies ist notwendiger Teil der Pluginstruktur von \jalgo.\\
Weiterhin hat die Informationseinheit das \textbf{Singleton}-Entwurfsmuster zu implementieren mit der Zugriffsmethode
\verb|public static IModuleInfo getInstance();|\\
Als Klassenmethode kann diese nicht in das Interface \verb|IModuleInfo| aufgenommen werden. Es sei jedoch ausdr�cklich darauf hingewiesen, dass das Fehlen dieser Methode dazu f�hrt, dass das Modul nicht korrekt erkannt wird und dass Laufzeitfehler beim Starten des Programmes auftreten.

Als Beispielcode ist ein minimalistisches Modul implementiert namens \verb|testModule|. Es ist ein korrekt implementiertes Modul, jedoch hat es keinerlei Funktionalit�t. Der Entwickler kann den Code bei Bedarf als Skelett zum Aufsetzen eines neuen Moduls nehmen. Zum aktuellen Zeitpunkt existieren f�r \jalgo 2 Module: \textbf{AVL-B�ume} und \textbf{Dijkstra}. Es sei dem Entwickler freigestellt, diese als Anleihe zu nehmen.\\[0.5cm]
Mit diesen beiden Klassen ist die modulseitige Schnittstelle fertiggestellt. Damit das Modul als solches auch vom Hauptprogramm erkannt wird, ist noch ein Schritt notwendig.

\subsection{Pluginstruktur von \jalgo}
An dieser Stelle scheint es angebracht, kurz die Pluginstruktur von \jalgo zu erl�utern. In der Distribution wird das Hauptprogramm in ein JAR-Archiv verpackt. Es muss unabh�ngig von den Modulen sein. Daher wird auch jedes Modul in ein eigenes JAR-Archiv verpackt mit allem, was zu diesem Modul geh�rt: Code und Ressourcendateien. Dies garantiert, dass bei Erscheinen eines neuen Modules nur das entsprechende JAR-Archiv vom Benutzer heruntergeladen werden muss.\\[0.25cm]
\textbf{Achtung!} Die JAR-Archive f�r die Module m�ssen als Namen das Modulk�rzel tragen und im Ordner \verb|runtime/modules| liegen. Nur so kann das Modul vom Hauptprogramm erkannt werden.\\[0.25cm]
W�hrend der Laufzeit wird zum Start des Hauptprogramms der Ordner \verb|runtime/modules| nach JAR-Archiven durchsucht. Dabei wird der Name des Archives als Paketname angenommen, und es wird in jedem Archiv nach den beiden Verbindungsklassen (siehe oben) gesucht:\\
\verb|org.jalgo.module.|\textsc{<Archivname>}\verb|.ModuleConnector| und\\
\verb|org.jalgo.module.|\textsc{<Archivname>}\verb|.ModuleInfo|\\
Sind diese korrekt implementiert, wird das Modul in die Liste der installierten Module aufgenommen und kann vom Benutzer ausgew�hlt werden.

Der Entwickler hat also nun noch eine leere Datei mit dem Namen \textsc{<Modulk�rzel>}\verb|.jar| im Ordner \verb|runtime/modules| zu erstellen. Wird \jalgo aus der Entwicklungsumgebung ge\-startet, kann das Modul nun aufgerufen werden.
\newpage

\subsection{Organisation der Ressourcen}
Sicher soll das Modul irgendwelche Ressourcendateien halten, wie zum Beispiel Icons oder ausgelagerte Algorithmentexte.

Da, wie erw�hnt, zur Laufzeit das Modul in einem JAR-Archiv vorliegt, kann auf die Ressourcen nur �ber den Klassenlader zugegriffen werden, indem die Methoden\\ \verb|getClass().getResource(String)| (liefert eine URL) oder\\ \verb|getClass().getResourceAsStream(String)| (liefert einen InputStream)\\
verwendet werden. Es erweist sich als vorteilhaft, wenn die Pfade zu den Ressourcendateien nicht direkt im Code verankert werden, sondern in einer externen Textdatei abgelegt werden. Um einen einfachen Ressourcenzugriff zu erm�glichen, bietet das Hauptprogramm mit der Klasse \verb|org.jalgo.main.util.Messages| die Methode\\
\verb|getResourceURL(String bundleKey, String key)|\\
an, welche direkt die URL einer Ressource zur�ckgibt. Der erste Parameter ist der Schl�ssel, mit welchem das Ressourcenpaket ausgew�hlt wird, aus dem der Pfad zu entnehmen ist. Dies ist wieder das Modulk�rzel, also der Hauptpaketname, wenn moduleigene Ressourcen geladen werden sollen, und \verb|main|, wenn Ressourcen des Hauptprogramms, z.B. Standard-Icons verwendet werden sollen. Der zweite Parameter ist der Schl�ssel der Ressource. Daf�r ist im Hauptpaket des Moduls eine Textdatei zu erstellen, in welcher Ressourcenpfade zu Schl�sseln zugeordnet werden. Es sei als Beispiel auf die existierenden Dateien von Hauptprogramm und den bestehenden Modulen verwiesen.\\[0.25cm]
\textbf{Achtung!} Die Textdatei mit den Ressourcenpfaden muss einer Namenskonvention folgen: Sie hat den Titel \verb|res.properties| zu tragen und muss im Hauptpaket des Moduls liegen. Anderenfalls wird sie von der Klasse \verb|Messages| nicht gefunden.\\[0.25cm]
Die Ressourcendateien selbst werden im Ordner \verb|res/module/|\textsc{Modulk�rzel} abgelegt. Wird unter Eclipse programmiert, ist dieser Ordner unter "`Projekteinstellungen"'->"`Libraries"'-> \mbox{"`Add Class Folder..."'} hinzuzuf�gen, damit die Ressourcen in der Entwicklungsumgebung freigegeben sind.\\
Um Namenskonflikten unter den Ressourcendateien vorzubeugen (letzlich liegen alle Ressourcenpfade hinter \verb|res/main/| und \verb|res/module/|\textsc{<Modulk�rzel>}\verb|/| auf dem Klassenpfad), empfiehlt es sich, im angelegten Ressourcenordner eine angemessene Struktur zu entwickeln, so zum Beispiel einen Unterordner \textsc{<Modulk�rzel>}\verb|_pix| f�r Bilddateien. Jetzt kann auch ein Dateiname wie \verb|icon.gif| ohne Probleme verwendet werden.\vspace{0.5cm}

\jalgo ist ein Programm, welches mehrere Sprachen unterst�tzt. Zum aktuellen Zeitpunkt sind s�mtliche Programmteile in Deutsch und Englisch verf�gbar. Dem Entwickler wird nahegelegt, auch das neue Modul in diesen Sprachen zu ver�ffentlichen. Dazu ist es notwendig, alle Zeichenketten, die dem Benutzer dargeboten werden sollen, in externen Textdateien zu speichern. Auch f�r den einfachen Zugriff auf diese Zeichenketten bietet die Klasse \verb|org.jalgo.main.util.Messages| eine Methode an:\\
\verb|getString(String bundleKey, String messageKey)|\\
Die Verwendung dieser Methode erfolgt analog zu der oben erw�hnten Methode f�r die Ressourcen.\newpage
\textbf{Achtung!} Die Textdatei mit den ausgelagerten Zeichenketten muss einer Namenskonvention folgen: Auch sie hat die Endung \verb|.properties| zu tragen. Der Name der Datei ist einfach das K�rzel der Sprache, f�r welche sie Zeichenketten enth�lt. F�r deutsch also \verb|de.properties|, f�r englisch \verb|en.properties|. Auch diese Textdateien haben im Hauptpaket des Moduls zu liegen.\\[0.25cm]
Liegen die Textdateien korrekt vor, so wird vom Hauptprogramm automatisch auf die eingestellte Sprache umgestellt. Der Modulentwickler muss hierzu nichts mehr beachten.\\
Auch hier wieder sei als Beispiel auf die existierenden Dateien von Hauptprogramm und Modulen verwiesen.\vspace{0.5cm}

Letzlich sei noch erw�hnt, dass jedes Modul die M�glichkeit hat, persistente Benutzereinstellungen anzubieten. Will der Modulentwickler solche Einstellungen einbauen, so muss eine Textdatei mit Zuordnungen zwischen Schl�sseln und Werten angelegt werden. Es sind hier alle Einstellm�glichkeiten als Schl�ssel zu erw�hnen; die zugeordneten Werte sind jeweils die Default-Werte f�r die entsprechenden Einstellungen.\\[0.25cm]
\textbf{Achtung!} Die Textdatei mit den Default-Einstellungen muss einer Namenskonvention folgen: Sie muss den Namen \textsc{<Modulk�rzel>}\verb|.prefs| tragen und im Ressourcenverzeichnis des Moduls (\verb|res/|\textsc{<Modulk�rzel>}\verb|/|) liegen. Au�erdem muss ein Schl�ssel namens "`Version"' in der Datei stehen, dessen Wert die Version der Einstellungsdatei angibt. Wird in einer sp�teren Version etwas am Bestand der Schl�ssel ge�ndert, gew�hrleistet der ge�nderte Wert von "`Version"', dass beim Benutzer die Einstellungsdatei neu erstellt wird und an die neuen Einstellungen angepasst wird.\\[0.25cm]
Es wird als Beispiel auf die Dateien \verb|main.prefs| des Hauptprogramms und \verb|avl.prefs| des Moduls AVL-B�ume verwiesen.\\
Zugegriffen wird auf die persistenten Einstellungen �ber die Klasse\\ \verb|org.jalgo.main.util.Settings|\\
F�r Details ist die API-Dokumentation von \jalgo zu konsultieren. Nachfolgend ist der relevante Teil von \verb|Settings| abgebildet:
\begin{verbatim}
public class Settings {

    public static boolean getBoolean(String resourceKey, String settingKey);
    public static void setBoolean(String resourceKey, String key,
        boolean value);
    public static String getString(String resourceKey, String key);
    public static void setString(String resourceKey, String key,
        String value);
}
\end{verbatim}
Wie schon beim Zugriff auf Ressourcenpfade und ausgelagerte Zeichenketten gibt der erste Parameter jeder Methode hier an, in welcher Einstellungsdatei die Einstellm�glichkeit gesucht werden soll. Dies ist f�r modulspezifische Einstellungen wieder das Modulk�rzel. Es ist aber auch m�glich, an die Einstellungen des Hauptprogramms zu gelangen mittels des \verb|resourceKey|s \verb|main|.
\newpage

\subsection{Organisation der Hilfe-Dateien}
Dem Entwickler wird angeraten, ebenfalls eine Online-Hilfe zu seinem Modul zu erstellen. Die Online-Hilfe von \jalgo nutzt die Technologie von JavaHelp 2.0. An dieser Stelle kann lediglich ein kleiner Einblick in den Umfang dieses System erm�glicht werden, und es wird vor allem auf die \href{http://java.sun.com/products/javahelp/}{Support-Seiten von Sun} und \href{http://www.knopf.com/resources/help/javahelp.html}{Tutorials} verwiesen.\\
Es folgen 5 Abschnitte. Der erste gibt einen groben �berblick �ber die f�r JavaHelp notwendigen Ressourcen. Die Teile 2 bis 4 befassen sich dagegen mit einzelnen Ansichten, mit denen der Zugriff auf die Hilfetexte angenehmer gestaltet werden kann. Zum Schluss wird im letzten Abschnitt noch auf die Einbettung der Hilfe in \jalgo eingegangen.\\
\textbf{Hinweis!} Beispieldateien zur Hilfe sind in den Ressourcen des Hauptprogrammes im Ordner \verb|res/main/help| zu finden. Hier finden sich auch CSS Stylesheets und die config-Datei f�r die Volltextsuche.

\subsubsection{�berblick}
Bei JavaHelp handelt es sich um eine Betriebssystem-unabh�ngige Technologie, die in einem Java-basierten Browser HTML-Dateien anzeigen kann. Sie bietet die M�glichkeit, �ber Ansichten wie Inhaltsverzeichnis, Index�bersicht, Favoriten und Volltextsuche den Zugriff auf die Hilfe zu beschleunigen. Grundlegend f�r eine mit JavaHelp konstruierte Hilfe sind zwei Dateien.
\begin{itemize}
	\item Die erste ist das sogenannte \textbf{HelpSet}. Diese Datei vereinigt alle f�r die Hilfe wichtigen Daten und dient der Applikation als eine Art Kontaktdatei. Hier wird unter anderem festgelegt, welche Ansichten die Hilfe unterst�tzen soll und welche Dateien daf�r benutzt werden. Au�erdem wird hier die Map-Datei festgelegt.
	\item Die \textbf{Map-Datei} ist ein Liste von Schl�ssel-URL-Zuordungen. Alle von der Hilfe ben�tigten Resourcendateien, wie HTML- oder Bilddateien, werden hier aufgelistet und einem Schl�ssel zugeordnet. In den Dateien f�r die einzelnen Ansichten werden anschlie�end nur noch Schl�ssel als Verweis auf bestimmte Ressourcen verwendet.
\end{itemize}
Neben diesen beiden wichtigen Metadateien besitzt jede Ansicht eine oder mehrere beschreibende Dateien. Dazu jetzt mehr.

\subsubsection{Inhaltsverzeichnis}
Das Inhaltsverzeichnis (Table Of Contents) erm�glicht einen schnellen inhaltsbezogenen Zugriff auf spezielle Hilfethemen und ist daher ungemein praktisch. Aus diesem Grund ist es f�r eine Hilfe unersetzlich. Die Struktur des Verzeichnisbaumes wird in einer XML-Datei beschrieben, die traditionell den Namen \textsc{<Modulk�rzel>}\verb|TOC.xml| tr�gt. Die genaue Struktur einer solchen Datei l�sst sich am besten am Beispiel der Hilfedateien des Hauptprogramms und der bestehenden Module ergr�nden und wird hier nicht weiter beleuchtet.

\subsubsection{Index�bersicht}
Die Index�bersicht gew�hrt �hnlich dem Inhaltsverzeichnis einen �berblick �ber die verschiedenen Hilfethemen. Doch ist sie im allgemeinen alphabetisch geordnet. Auch die Index�bersicht wird durch eine XML-Datei beschrieben. Traditionell ist diese mit dem Namen \mbox{\textsc{<Modulk�rzel>}}\verb|Index.xml| betitelt. Auch hier sei bez�glich der Struktur dieser Datei ein Blick in die bestehenden Dateien der Module und des Hauptprogramms ans Herz gelegt.

\subsubsection{Volltextsuche}
Die Volltextsuche ist ein sehr hilfreiches Feature, welches es erlaubt, den gesamten Text der Hilfe nach bestimmten W�rtern zu durchsuchen. Das Erstellen einer solchen Suchoption gestaltet sich nicht so einfach wie bisher und es werden mehrere Dateien daf�r verwendet, die im allgemeinen in einem Ordner namens \verb|JavaHelpSearch| des JavaHelp-Ordners abgelegt werden. Im HelpSet wird zu dem auch nur auf diesen Ordner verwiesen und nicht auf eine bestimmte Datei.\\
JavaHelp bietet f�r die Erzeugung einer Suchengine einen Konsolenbefehl, der alle notwendigen Dateien erzeugt. Der sogenannte \textbf{jhindexer}-Befehl wird im Ordner ausgef�hrt, in dem das HelpSet zu finden ist. Er durchl�uft die HTML-Dateien und vergibt f�r jedes Wort einen Index. Sp�ter wird mit Hilfe dieser das Wort wieder gefunden. Nun ist es ratsam bestimmte W�rter, wie "`eine"', "`keine"', "`mit"' oder "`oder"', aus der Indexierung auszuklammern. Diese sogenannten \textbf{Stopwords} werden in einem config-File aufgelistet und \textbf{jhindexer} �bergeben.\\[0.25cm]
Der Befehl hat dann die folgende Form:\\
\verb|jhindexer -c|\textsc{ <config-Datei> <Pfad der HTML-Dateien>}\\[0.25cm]
ein Beispiel: \\
\verb|jhindexer -c|\textsc{ /JavaHelpSearch/config.ini ../html/data}\\[0.25cm]
\textbf{Hinweis!} Der jhindexer-Befehl ist weitaus umfangreicher und es wird empfohlen, alles weitere im JavaHelpUserGuide (jhug.pdf) nachzuschlagen.

\subsubsection{Einbettung von JavaHelp in \jalgo}
Nachdem nun ein wesentlicher �berblick �ber die Dateien und Ressourcen von JavaHelp entstanden sein sollte, wird im kommenden Abschnitt auf die Einbettung dieser in die Umgebung von \jalgo eingegangen.\\
Grundlegend besteht keine Namenskonvention, die unbedingt eingehalten werden muss. Nur empfiehlt es sich, ausdrucksstarke Namen sowohl f�r die Datei des HelpSets als auch f�r Map- und Ansichtsdateien zu verwenden. Das HelpSet sammelt alle wichtigen Dateinamen, und es ist nur ausschlaggebend, wie diese Datei benannt wird. Die bestehenden Module und das Hauptprogramm folgen hier der folgenden Namenskonvention: \textsc{<Modulk�rzel>}\verb|_help.hs|.\\
Um die Hilfe schlie�lich aus dem \jalgo - Hauptprogramm aufrufbar zu machen, wird die Methode \verb|org.jalgo.module.|\textsc{<Modulk�rzel>}\verb|.ModuleInfo.getHelpSetURL()| genutzt. Diese liefert den Pfad der HelpSet-Datei. Da es sich bei dem HelpSet um eine Ressource von \jalgo handelt, wird empfohlen, die Methode \verb|getResourceURL(String bundleKey, String key)| der Klasse \verb|org.jalgo.main.util.Messages| wie beschrieben zu nutzen.

\newpage
\subsection{Schnittstelle zum Hauptprogramm}
Die zweite erw�hnte Schnittstelle ist die auf Seiten des Hauptprogramms, namentlich die Klasse \verb|org.jalgo.main.gui.JAlgoGUIConnector|. Der Entwickler hat hier nichts zu implementieren, jedoch hat er Kenntnis von dieser Schnittstelle zu haben. Hier�ber laufen alle Anfragen, die das Modul an die graphische Oberfl�che des Hauptprogramms richtet. F�r Details dazu sei auf die API-Dokumentation von \jalgo verwiesen. Nachfolgend ist die Schnittstelle von \verb|JAlgoGUIConnector| abgebildet. Gerade die letzten 3 Methoden sind interessant, um an die moduleigenen GUI-Komponenten zu gelangen.
\begin{verbatim}
public class JAlgoGUIConnector {

    public static JAlgoGUIConnector getInstance();

    public void saveStatusChanged(AbstractModuleConnector moduleInstance);
    public void showErrorMessage(String msg);
    public void showWarningMessage(String msg);
    public void showInfoMessage(String msg);
    public int showConfirmDialog(String question, int optionType);
    public void setStatusMessage(String msg);
    public String showOpenDialog(boolean openAsJAlgoFile,
                                 boolean useCurrentModuleInstance);
    public AbstractModuleConnector newModuleInstanceByName(String moduleName);

    public JComponent getModuleComponent(AbstractModuleConnector module);
    public JMenu getModuleMenu(AbstractModuleConnector module);
    public JToolBar getModuleToolbar(AbstractModuleConnector module);
}
\end{verbatim}
Diese Klasse implementiert das \textbf{Singleton}-Entwurfsmuster. Somit kommt man �ber die Zugriffsmethode \verb|getInstance()| an die Instanz.
\newpage
\section{Bekannte Fehler und Schwachstellen}
Derzeit wird vom Hauptprogramm kein Drucken unterst�tzt. Daher ist die Methode\\
\verb|AbstractModuleConnector.print()| derzeit nutzlos. Sie wird trotzdem aus Kompatibilit�tsgr�nden mitgef�hrt, um f�r zuk�nftige Implementationen ger�stet zu sein.

\section{Weiterf�hrende Links}
Die Software nutzt f�r die graphische Oberfl�che die \textbf{Swing}-Technologie. Informationen hierzu findet man unter\\
\url{http://java.sun.com/docs/books/tutorial/},\\
\url{http://java.sun.com/products/jfc/}\\
\url{http://java.sun.com/products/javahelp/}\\
oder in der API-Dokumentation von Java.

\end{document}
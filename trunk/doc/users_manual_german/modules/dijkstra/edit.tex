\section{Graph erstellen}
Nach dem Start des Moduls wird die Oberfl�che f�r das Erstellen eines Graphen angezeigt. Sie ist
in die Bereiche
\begin{enumerate}
	\item "`Werkzeuge"'
	\item "`Graph"'
	\item "`Knotenliste"'
	\item "`Kantenliste"'
	\item "`Distanzmatrix"'
\end{enumerate}
aufgeteilt.

\subsection{Graphische Eingabe/Erstellen eines Graphen per Maus}
Das Erstellen eines Graphen per Maus wird durch die Werkzeuge
\begin{enumerate}
	\item "`Knoten hinzuf�gen/verschieben"' --- Durch Klicken auf die Zeichenfl�che wird ein neuer Knoten erzeugt. Ein bestehender Knoten kann durch Ziehen mit der Maus bewegt werden.
	\item "`Kante hinzuf�gen/bewerten"' --- Indem man die Maus von einem Knoten zu einem anderen zieht, entsteht zwischen ihnen eine neue Kante. Die Kantenbewertung wird ge�ndert, wenn man sie herauf- bzw. hinunterzieht.
	\item "`Knoten l�schen"' --- Ein angeklickter Knoten wird gel�scht.
	\item "`Kante l�schen"' --- Eine angeklickte Kante wird gel�scht.
\end{enumerate}
unterst�tzt.
Dabei geht man wie folgt vor: Nach der Auswahl des Werkzeugs "`Knoten hinzuf�gen / verschieben"' kann man durch einfaches Klicken auf die wei�e Zeichenfl�che Knoten erstellen. Vorhandene Knoten k�nnen mit Drag\&Drop verschoben werden.\\
Nachdem man alle Knoten angelegt hat, kann man nach Auswahl des Werkzeugs "`Kante hinzuf�gen/bewerten"' den Graphen vervollst�ndigen. Um eine Kante zu erstellen, klickt man erst den "`Startknoten"' und dann den "`Endknoten"' der Kante an. Es erscheint eine Kante zwischen den Knoten mit der Bewertung f�nf. Diese Bewertung (auch Kantengewicht) kann ver�ndert werden, indem man das Kantengewicht mit der Maus "`festh�lt"' und nach oben (das Gewicht wird gr��er) oder unten (das Gewicht wird kleiner) zieht.
\newpage
\centerpic{dijkstra/edit}{0.5}{Graphisches Erstellen eines Graphen}

\subsection{Die Knotenliste}
Die Knotenliste zeigt die Indizes aller Knoten durch Kommata getrennt. Durch Hinzuf�gen von
Indizes werden auch neue Knoten erzeugt. �nderungen in der Knotenliste werden nach Bet�tigen der Schaltfl�che "`Anwenden"' �bernommen.

\subsection{Die Kantenliste}
Die Kantenliste zeigt alle Kanten des Graphen im Format ( VON, WEG, ZU ). Durch Editieren
dieser Liste k�nnen bestehende Kanten ge�ndert und neue hinzuf�gt werden. Auch hier ist zu
beachten, dass �nderungen erst durch Klicken von "`Anwenden"' �bernommen werden.

\subsection{Die Adjazenzmatrix}
Die Adjazenzrelation des Graphen ist in dieser Matrix dargestellt. Kanten und Knoten k�nnen in jener durch einfaches Eingeben einer Kantenbewertung erzeugt werden. Dabei muss nicht beachtet werden, da� die Matrix symmetrisch bleibt, da dies automatisch gew�hrleistet wird.\\
Nach dem Bearbeiten der Matrix darf nicht vergessen werden, den "`Anwenden"' - Button zu bet�tigen, damit die �nderungen �bernommen werden.

\newpage